% !TeX spellcheck = da_DK
\documentclass[11pt, fleqn]{article}
%\usepackage{siunitx}
\usepackage{SpeedyGonzales}
\usepackage{MediocreMike}
\title{AI og Etik}
\author{Oskar Eiler Wiese Christensen s183917}
\date{}

\begin{document}
	
	\maketitle
	
	\section*{Etik}
	Et ord man bruger om diskussionen af korrekt opførsel. \\ Og hvad bør vi ikke gøre / ikke have gjort. \\ En filosofisk disciplin der beskæftiger sig med hvordan vi bør handle. \\ 
	
	\subsubsection*{Nyttetisk princip:}
	\vspace*{-0.2cm}
	Man bør handle således at det maksimerer samlet velfærd.
	
	\subsubsection*{Rettighedsprincip:}
	\vspace*{-0.2cm}
	Individer har umistelige rettigheder, f.eks. til ikke udelukkende at blive brugt som middel, privathed, osv.

	\subsubsection*{Teknologi: pros and cons}
	\vspace*{-0.2cm}
	Pro: Global mobilitet \\ Cons: "Global opvarmning"
	
	\subsubsection*{Value alignment problem}
	\vspace*{-0.2cm}
	We may not specify our objectives in such a complete and well-callibrated fashion that a machine cannot find an undesireable way to achieve them.
	
	\subsubsection*{Hvad skal man være opmærksom på?}
	\vspace*{-0.2cm}
	Ikke-intenderede bivirkninger -- ex klimaændringer \\
	Dual-use - hvordan kan teknologien bevidst bruges forkert? \\ 
	Økonomiske interesser - hvem tjener penge og har interesser i at forstyrre debatten om teknologien \\ 
	Politiske interesser
	
	\subsubsection*{Særlige problemstillinger}
	\vspace*{-0.2cm}
	Overvågning og uetisk meningspåvirkning \\
	Algoritmer der diskriminerer eller bliver brugt uetisk \\
	Ansvarsforflygtigelse i forhold til beslutninger \\ 
	Automatisering skaber og fjerner jobs 
	
	\subsubsection*{Løsninger}
	\vspace*{-0.2cm}
	Lovgivning omkring AI (f.eks. GDPR) \\
	Råd og komiteer (etiske råd)\\
	Forskning (Safe AI, Explainable AI, Responsible AI, Mchine Ethics) \\
	Folkelig modstand eller opbakning (civil ulydighed)
	
	\section*{safe AI}
	
	\subsubsection*{Utopi: Build AI for an effective society - respect human autonomy (respekt for individet)}
	\vspace*{-0.2cm}
	Without trust - no acces to personal data \\
	Without acces to personal data - no AI \\
	Withouth AI - no personalized services \\\\
	\noindent
	What is safe AI? -- Installing trust in intelligent systems. \\ Focus: How we can teach AI to keep secrets? - "privacy by design" \\ 
	State of the art: K-anonymity, multiparty secure computing 
	
	\subsubsection*{Personalized services require that we share private data}
	\vspace*{-0.2cm}
	Human variability - can be handled by individualised models \\
	- Powerlaws of behavior \\
	- Learning: The power law of practise \\ 
	- Standard medical solutions based on population effects (Need for a science of the individual to predict and act personally)
	\\\\
	Most behavioral information is "hidden" (incl. for owner) =$ > $ neurotechnology.\\
	- Brain: Motivation, vigilance, attention \\
	- History: Knowledge graphs, experience \\
	- Social dimensions: Network, sentiments
	\\\\
	(\textit{"freedom evolves" - bog om fri vilje})
	(Det at man ens egen bevidsthed lyver over for sig selv, f.eks. i forhold til rygning er godt, osv. - man skal sikre at bevidsthed man implementerer i AI er objektiv.)
	
	\subsubsection*{How can society gain trust in AI}
	\vspace*{-0.2cm}
	Safe AI = secure - test og verified software and hardwate, adversarials\\
	Safe AI = open source - methods, code, hardware, check and evolution \\
	Safe AI = self-conscious - understand own role \\
	Safe AI = can keep a scret - privacy by design \\
	Safe AI = has calibrated values - debug for stereotupes, biases \\
	Safe AI = is accountablee - transparent, communicating, "right to explanation" \\
	Safe AI = understands social relations - understands user's knowledge graph \\
	Safe AI = understands power - digital self-defense \\
	Safe AI = generates trust
	\\
	\subsection*{Privacy technology}
	\noindent
	\subsubsection*{Multi-party secure computing} 
	\vspace*{-0.2cm} \noindent
	Machine learning, Linear algebra in the encrypted domain (homomorphic cryptography) can evaluate algorithms on data without sharing. \\ \textbf{Solves}: shared computation between untrusting collaborators \\ 
	
	\subsubsection*{Differential privacy (algoritmer uden om modellen der estimere en "privacy" som er en parameter man kan træne netværk efter)} "Enbaled machine learning" \\
	Randomized ML with crypto guarantees against privacy attacks \\
	Many mechanisms proposed (ways of adding noise) \\
	Quanitative trade-off between performance and privacy \\
	\textbf{Solves}: attacks on individual (or group) privacy in machine learning systems
	
	\subsubsection*{AI is necessary}
	\vspace*{-0.2cm}
	Individual variability (power laws) \\
	Life, society, history, etc. are complex graphs to navigate 
	
	\subsubsection*{Safe AI is possible}
	\vspace*{-0.2cm}
	Many dimensions have solutions sketched \\
	socials compentences, attention, selfmonotoring \\
	
	\section*{Digitale spor og privatlivets fred}
	
	\subsection*{jeg har lavet noget FORSKNING se på mig NU!}
	\vspace*{-0.2cm}
	\subsection*{imdb = internet movie data base}
	\vspace*{0.4cm}
	
	 
	\section*{Etik, AI og Robotter}
	
	\subsection*{Etik}
	\vspace*{-0.2cm}
	Grundlag for hvordan det er at opføre sig korrekt \\
	Konflikten mellem nytteetisk princip og rettighedsprincip kan lede til etiske dilemmaer \\
	Normal opfattelse af at optimere velfærd er at optimere nydelsen og minimere smerten \\ 
	
	 \subsection*{Digitalisering og etik}
	 \vspace*{-0.2cm}
	 \textbf{Digitalisering giver anledning til særlige problemstillinger}:\\
	 Overvågning og uetisk meningspåvirkning \\
	 Algoritmer der diskriminerer eller bliver brugt uetisk \\
	 Ansvarsforflygtigelse i forhold til beslutninger \\
	 Automatisering skaber og fjerner jobs
	 \\
	 \subsection*{Etiske teorier og handlingsniveau}
	 \vspace*{-0.2cm}
	 \textbf{Utilitarisme}: vælg den handling, der maksimerer velfærden \\
	 \textbf{Kants kategoriske imperativ (anden version)} "Handl, således at menneskeheden i din egen person såvel som i enhver anden person aldrig kun behandles som middel, men altid tillige som mål".
	 \\ \textbf{Dydsetik}: handl sådan som den dydige person ville gøre \\\\
	 \textbf{Dobbelt-effektsprincippet}: \\
	 $ \bullet $ Din handling er tilladt, hvis, \\
	 i) Handlingen selv er moralsk god eller neutral \\
	 ii) Den positive konsekvens er intenderet og den negative konsekvens er ikke intenderet. \\
	 iii) Den negative konsekvens bruges ikke som middel til at opnå den positive konsekvens \\
	 iiii) Der er proportionalt set tungtvejende grunde til at foretrække den positive konsekvens mens man tillader den negative konsekvens
	
	 \subsection*{Hannah Arendt - Friheden og den enkelte i centrum}
	\vspace*{-0.2cm}
	Man cannot be free if he does not know that he is subject to necessity, because his freedom is always won in his never wholly successful attempts to liberate himself from necessity.
	
	
	\subsection*{Magtens ydre og indre grænser}
	\vspace*{-0.2cm}
	\textbf{Demokratiets ydre grænser} sætter grænser for magtudøvelse udadtil - typisk national grænser
	
	\subsection*{Totalitær magt}
	\vspace*{-0.2cm}
	Samfund hvor staten ønsker at kontrollere alt. Accepterer ikke indre grænser for magten. Vil ikke anerkende, at der er ting staten ikke blander sig i. 
	
	\subsection*{Social Credit System (SCS)}
	\vspace*{-0.2cm}
	"in the words of the Chinese government, the goal of SCS is to allow the trustworthy to roam everywhere under heaven while making it hard dfor the discredited to take a single step.
	
\end{document}
