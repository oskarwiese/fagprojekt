% !TeX spellcheck = en_US
\documentclass[11pt, fleqn, titlepage]{article}
%\usepackage{siunitx}
\usepackage{texfiles/SpeedyGonzales}
\usepackage{texfiles/MediocreMike}
\newcommand{\so}[2]{{#1}\mathrm{e}{#2}}
% \geometry{top=1cm}
\usepackage{hyperref}
\usepackage{amsmath}
\usepackage{ragged2e}
\usepackage{booktabs}
\usepackage{lipsum}
\usepackage{csquotes}
\usepackage{longtable}
\hypersetup{
	colorlinks=true,
	linkcolor=blue,
	filecolor=magenta,      
	urlcolor=cyan,
}
\usepackage{subfig}
\usepackage{graphicx}
\title{Fairness in Classification}
\author{Anders Henriksen \\ Oskar Eiler Wiese Christensen  \\ \texttt{\{s183917, s183904\}@student.dtu.dk}}
\date{\today}

\pagestyle{plain}
\fancyhf{}
\rfoot{Page \thepage{} of \pageref{LastPage}}

\graphicspath{{Billeder/}}

\begin{document}
	
	\maketitle
	\begin{abstract}
		\textbf{TODO: Skal indeholde motivation, problem, fremgangsmåde, resultater og konklusion.} \\ \lipsum[1-2]
	\end{abstract}
	\tableofcontents \newpage
	%\thispagestyle{fancy}
	%\tableofcontents	
	
	\section{Abstract}
	\textbf{TODO: Skal indeholde motivation, problem, fremgangsmåde, resultater og konklusion.}
	
	\section{Introduction}
	\textbf{TODO: \\ Hvad er formålet med projektet? \\ Hvad er problemformuleringen \\ Hvad er state-of-the-art? \\ Hvilken fremgangsmåde bruges til at løse problemet? \\ Hvem har brug for resultaterne?}
	
	\subsection{Motivation}
	Artifical intelligence (AI) and machine learning (ML) methods are playing a bigger and bigger role in modern society. As the accuracies of AI and ML models increase, their applications become wider, allowing for these  models to be implemented either as ground truth or as a pointer in applications like autonomous vehicles, medical imaging, the American judicial system. These models are, for the general public, often seen as an objective decision maker. As such, it becomes of essence to avoid discrimination, since discrimination in models could lead to reinforced societal discrimination. Bias seems to originate from the dataset, meaning that a saving grace would be to either implement bias correction on the dataset to remove the bias from the source of the problem or to remove bias from the model, thereby removing the risk of discrimination when using the model for classification tasks. This report aims to understand bias and how it affects datasets as well as classification models. Furthermore, a set of bias correction methods will be put to use in order to explore the possibilites of keeping discrimination away from important fields. More specifically, the questions to be answered throughout the report are whether there is a bias in the COMPAS recidivism dataset and which kind of bias there is, as well as what this bias means in the dataset. Meanwhile, it is also analyzed how bias affects a classification algorithm, how bias in an algorithm is detected and how to quantify the bias. Lastly, bias correction methods will be implemented and tested and an ethical discussion will be carried out to understand the applications of bias correction algorithms and AI models on society as well as how society can learn to trust these models. 
	
	\subsection{State of the Art}
	There are currently two main ways to ensure fairness in classification. One is by implementing steps during the training process of the classifier, and second is post-processing. Many different suggestions of how to implement bias correction algorithms within the field of fair AI have been proposed. 
	
	An approach proposed by Zafar et al. \cite{Zafar}, suggests that a way to achieve fair classifiers is by using linear constraints on the co-variance between predicted labels and the value of features. This is a method that is implemented during the training process.
	
	The main method that will be demonstrated in this project is proposed by Hardt et al. \cite{equal_of_oppor}. The suggested method can achieve a non-discriminatory classifier by post-processing the estimators of probability for a discriminant classifier to learn thresholds in order to deal with sensitive features or groups in the data. The method requires that the classifier has information about the data at decision-time, which is the case in the experiment conducted in this project. The idea is to first train an unfair classifier as usual and then using post hoc correction to correct the classifier. 
	
	
	\subsection{Contributions}

	
	
	\section{Data}
	\textbf{TODO: \\ Data kan kort introduceres i indledningen \\ Lav etisk diskussion om opbevaring af data samt privacy issues}
	
	\noindent To analyse the efficacy of using bias correction to remove discrimination among races in the American justice system, the modified COMAS recidivism dataset from ProPublica has been used. \\
	A general overview of the dataset and how it came to be as well as the variables used for the classifier is given in \ref{dataDescription}. The data and variables have not been properly explained to any extent in examined literature, so \ref{dataExamination} will cover the meaning of all 53 variables of the dataset to remove ambiguity and set a common ground from which to base the future analysis. Lastly, \ref{dataVisuals} will show important visualizations of features of the dataset to cover which of the variables are most likely to contain biases that will be explored further in later sections.
	
	\subsection{Description of Data} \label{dataDescription}
	The data used in this project stems from an initial analysis of the COMPAS (Correctional Offender Management Profiling for Alternative Sanctions) algorithm by its developers, Northpointe Inc. After this analysis, ProPublica made a subsequent analysis of this data as well as their own queries of the offenders involved and data of the offenders who actually recidivated. This data is stored in the \texttt{compas-scores-two-year.csv} dataset from ProPublica's GitHub page, which can be found here: \url{https://github.com/propublica/compas-analysis/blob/master/compas-scores-two-years.csv}. \\\\
	\noindent The data consists of 53 different variables, 9 of which are used in the binary classification model. Four of the chosen variables are categorical, so these will have to be one-out-of-k encoded to be able to feed the necessary variables into the neural network. Meanwhile, the numerical variables, of which there are also four, will be normalized as to avoid the vanishing gradient problem and to avoid having some variables be of more importance to the final prediction. This will be explained more thoroughly in \ref{Feed-forward neural}. \\
	The 10 chosen variables, of which one will be used as the target variable are shown and explained below.
	
	
	\begin{table}[H]\label{resultater}
		\centering
		\begin{tabular}{l l l}
			Variable & Description & Type \\ \hline
			age & The age of the offenders & Continuous ratio \\
			priors\_count & The number of previous offences & Discrete interval \\
			juv\_fel\_count & The number of previous juvenile felonies & Discrete interval \\
			juv\_misd\_count & The number of previous juvenile misdemeanor & Discrete interval \\
			c\_charge\_degree & The severity of the offence & Discrete nominal \\
			race & The race of the offender & Discrete nominal \\
			age\_cat & The age category of the offender & Discrete nominal \\
			sex & The sex of the offender & Discrete nominal \\
			score\_text & The COMPAS prediction of chance of recidivism & Discrete interval
		\end{tabular}
		%\caption{text}
	\end{table}
		
		
	\subsection{Explanation of Data Variables} \label{dataExamination}
	To remove ambiguity about the meaning of the variables in the COMAS recidivism dataset, an exhaustive description of every variable and variable type has been produced. This allows for the results of this report to be contained within this interpretation and shines light on the true contents of the dataset, as no one else has done this before.
	
	
	\begin{longtable}{l l l}
		Variable & Description & Type \\ \hline
		id & Index of the column & Discrete interval \\
		name & Full name of the offender & Discrete nominal \\
		first & First name of the offender & Discrete nominal \\
		last & Last name of the offender & Discrete nominal \\
		compas\_screening\_date & The date the COMAS score was given & Discrete nominal \\
		sex & The sex of the offender & Discrete nominal \\
		dob & The offender's date of birth & Discrete nominal \\
		age & The offender's age & Continuous ratio \\
		age\_cat & Which age category the offender belongs to & Discrete nominal \\
		race & The offender's race & Discrete nominal \\
		juv\_fel\_count & The number of previous juvenile felonies & Discrete interval \\
		Decile\_score & Value from 1-10 representing risk of recidivism & Discrete ordinal \\
		juv\_misd\_count & The number of previous juvenile misdemeanor & Discrete interval \\
		juv\_other\_count & The other types of previous juvenile crimes & Discrete interval \\
		priors\_count & The number of previous offences & Discrete interval \\
		days\_b\_screening\_arrest & ----- & Discrete interval \\
		c\_jail\_in & The date the offender was placed in custody & Discrete nominal \\
		c\_jail\_out & The date the offender was removed from custody & Discrete nominal \\
		c\_case\_number & The custody case number & Discrete nominal \\
		c\_offense\_date & ----- & Discrete nominal \\
		c\_arrest\_date & ----- & Discrete nominal \\
		c\_days\_from\_compas & ------ & Discrete interval \\
		c\_charge\_degree & The severity of the offence (misdemeanor or felony) & Discrete nominal \\
		c\_charge\_desc & The offense that was performed & Discrete nominal \\
		is\_recid & Whether the offender truly recidivated after two years (0/1) & Discrete nominal \\
		r\_case\_number & The recidivism case number & Discrete nominal \\
		r\_charge\_degree & The recidivism offense severity & Discrete nominal \\
		r\_days\_from\_arrest & Recidivism days until arrested for the crime & Discrete interval \\
		r\_offense\_date & The date the recidivism offense was performed & Discrete nominal \\
		r\_charge\_desc & The recidivism crime commited & Discrete nominal \\
		r\_jail\_in & The date the offender was jailed & Discrete nominal \\
		r\_jail\_out & The date the offender was removed from jail & Discrete nominal \\
		violent\_recid & Completely empty data column & NaN \\
		is\_violent\_recid & Whether the offender truly recidivated violently after two years (0/1) & Discrete nominal \\
		vr\_case\_number & The violent recidivism case number & Discrete nominal \\
		vr\_charge\_degree & The violent recidivism offense severity & Discrete nominal \\
		vr\_offense\_date & The date the violent recidivism was performed & Discrete nominal \\
		vr\_charge\_desc & The violent recidivism crime commited & Discrete nominal \\
		type\_of\_assessment & What the COMPAS algorithm predicted for the offender & Discrete nominal \\
		decile\_score\_1 & ------ & Discrete ordinal \\
		score\_text & decile\_score split into three categories & Discrete interval \\
		screening\_date & When the offender was given the assessment & Discrete nominal \\
		v\_type\_of\_assessment & COMPAS violent prediction type & Discrete nominal \\
		v\_decile\_score & ------ & Discrete ordinal \\
		v\_score\_text & v\_decile\_score split into three categories & Discrete interval \\
		v\_screening\_date & When teh offender was given the assessment & Discrete nominal \\
		in\_custody & When the offender was placed in custody & Discrete nominal \\
		out\_custody & When the offender was removed from custody & Discrete nominal \\
		priors\_count\_2 & The exact same column as priors\_count & Discrete interval \\
		start & The day the screening started compared to the start of the process & Discrete interval \\
		end & The day the screening ended compared to the start of the process & Discrete interval \\
		event & ------ & Discrete interval \\
		two\_year\_recid & Whether the offender truly recidivated (violent or non-violent) after two years & Discrete interval
	\end{longtable}
		
	\subsection{Visualization of Data} \label{dataVisuals}
	To better get an understanding of the data, a range of plots are shown below. \ref{fig:predictedrecidrace} aims to show wheter there is a difference between the fraction of whites and african-americans that are predicted by the COMPAS classifier as having low, medium and high risk of recidivism. \ref{fig:predictedrecidsex} has the same purpose, utilizing sex instead of race. \ref{fig:truerecid} and \ref{fig:proirs} give some insight into the actual amount of crime done by african-americans and whites and whether the amounts seem to be similar.
	
	\begin{figure}[H]
		\centering
		\includegraphics[width=0.5\linewidth]{imgs/predicted_recid_race}
		\caption{It is clear from this illustration that the number of caucasian and african-american people in the low group are almost equal. Meanwhile the medium and high groups contain a much larger fraction of african-americans. Since, in total, there are more african-americans in the dataset than whites (3696/2454), it would seem based purely on the data that whites are more often classified as low risk of recidivism while african-americans are often classified as medium or high risk.}
		\label{fig:predictedrecidrace}
	\end{figure}
	
	\begin{figure}[H]
		\centering
		\includegraphics[width=0.5\linewidth]{imgs/true_recid}
		\caption{This illustration shows the true recidivism values (0 being no recidivism after two years and 1 being recidivism) for all offenders in the dataset seperated by the race of each offender. It is clear that close to an equal amount of white and african-american offenders did not recidivate, while a much larger proportion of african-americans re-offended than whites. This makes it difficult to prove bias in the data based only on the data, as it is not necessarily a bias that african-americans more often re-offend.}
		\label{fig:truerecid}
	\end{figure}
	
	\begin{figure}[H]
		\centering
		\includegraphics[width=0.5\linewidth]{imgs/predicted_recid_sex}
		\caption{This illustration shows the relationship between sex and the COMPAS prediction of recidivism. The first point to be noted is that there are a lot more men in the dataset than women (5819/1395). Aside from this, it does not seem like there is a difference in the proportions of men or women being classified as belonging to each of the categories.}
		\label{fig:predictedrecidsex}
	\end{figure}
	
	\begin{figure}[H]
		\centering
		\includegraphics[width=0.5\linewidth]{imgs/proirs}
		\caption{An illustration of the relation between race (african-american or caucasian) and the number of previous felonies prior to the study taking place. This plot gives a clear indication that the number of whites and african-americans who have performed no prior felonies is similar. Otherwise, it is clear that the proportion of african-american to whites becomes larger as the number of previous felonies becomes larger. This seems to indicate that african-americans are more often engaged in criminal activity, which also weakens the indication of a bias from \ref{fig:predictedrecidrace}.}
		\label{fig:proirs}
	\end{figure}
	
	
	
	\subsection{Data ethicality}
	The COMPAS dataset as well as ProPublica's expanded data and analysis of this data is freely available from their public GitHub repository. As such, the data is not stored safely. This is also true for the variables of the data such as age, full name and number of previous felonies, which give ample opportunity to identify people who have recidivated, leading to compromise of their personal data. This is not a concern though, as the data is collected from Florida, where public records are subject to the broad legislated public right of inspection. According to chapter 119 section 1 of the law of the State of Florida, 
	\begin{displayquote}
		"It is the policy of this state that all state, county, and municipal records are open for personal inspection and copying by any person. Providing access to public records is a duty of each agency." \cite{floridaLaw}
	\end{displayquote}
	
	\noindent As such, since ProPublica submitted a public records request for access to the data for use in research \cite{propublicaAnalysis}, which inherently is a type of inspection, they are within the legal ramifications to use the dataset as they please. Another debate entirely, which will in particular be taken up in \ref{discussion}, is the more subjective question of the ethicality of open-record laws and public criminal data in their entirety being shared among whomever and what exactly should be allowed to be done with this data.
	
	\subsection{Bias in Data}
	
	
	\section{Methods}
	\textbf{TODO: \\ referér til kode og software \\ brug lang tid på nye metoder, kort tid på gamle metoder}
	
	\subsection{Binary Classifier}\label{Feed-forward neural}
	%initialization schemes
	\subsubsection{Feed-forward Neural Network}
	Feed-forward neural networks (FFNN) are the simplest form of a neural network. Information flow in a feed-forward neural is one directional which means that the network has an input layer, and information flow from these nodes through the hidden layers unto the output layer. The main purpose of a FFNN is to approximate a function. In this project $ y = f^*(x) $ maps an input $ \mathbf x $ to a category $ \mathbf y $. The FFNN is thereby a classifier, which goal is to determine the recidvism risk of a person given the input variable $ \mathbf x $. The input variable $ \mathbf x $ contains :::::. The categories of $ \mathbf y $ is either 0 or 1, which corresponds to the classifier classifying a person as either low or medium/high risk of recidivism. Hence, this is a binary classifier which constructs a mapping $ \mathbf y = f(\mathbf x \ ; \mathbf w) $. The goal is to learn the weights that most efficiently approximate the function $ f^* $ through supervised learning. The binary classifier in this project has an input layer, three hidden layers and an output layer. 
	
	\begin{figure}[H]
		\centering
		\includegraphics[width=0.5\linewidth]{imgs/ffnn}
		\caption{Visualization of the binary classifier model. The input layer has dimension $ \mathbb R ^9$ and the output layer has dimension $ \mathbb R^2 $. The number of nodes in the hidden layers is a changeable parameter which is found with Bayesian Optimization. }
		\label{fig:ffnn}
	\end{figure}
	
	
	\subsubsection{Bayesian Optimization}
	%Bayesian Optimization
	
	To obtain the pinnacle of accuracy in a neural network, it needs an excellent architecture as well as the potimal hyperparameters. However, the search for these parameters are usually a costly process of uncertainty balancing exploration of parameters and the exploitation of results. Often the process of finding the optimal parameters and network architecture comes from expert knowledge, biased heuristics or exhaustive sampling from the parameter space. To determine the architecture and hyperparameters of the binary classifier Bayesian Optimization is used. 
	\subparagraph*{Objective Function}
	The objective function that is being optimized in this project is the binary classifier presented above. The fully connected layers are trained on the COMPASS data-set and the validation accuracy of the model is what BO optimizes. The accuracy of the constructed model is evaluated using non-parametric Gaussian Process (GP) as a function fo the following hyperparameters:
	\begin{itemize}
		\item Number of units in the first hidden layer \(\in [1, 5000]\) 
		\item Number of units in the second hidden layer \(\in [1, 5000]\) 
		\item Number of units in the third hidden layer \(\in [1, 5000]\)
		\item Dropout probability \(\in [0,1]\)
		\item Activation function \{tanh, ReLU, ReLU6, Sigmoid\}
	\end{itemize}
	The hyperparameters that are not changed are the learning rate, which is set at $ \alpha = 0.001 $. The reason why this parameter is not changed is due to the fact, that the optimizer implemented in this project, is Adaptive Moment Estimation (Adam). Adam is a well known optimizer in the literature and has adaptive learning rate as well as step size. Which is why this learning rate is not an interchangeable hyperparameter. \\ Cross Entropy Loss is used as the cost-function. \textbf{TODO: skriv noget om funktionen. }
	\\\\
	Gaussian Process uses the smoothness assumption on the outputs of the network, and its variance and lengthscale parameters are optimized using maximum log-likelihood principle. \cite{aktiv_gp} The acquisition function, that choose the parameters for the model, used is Expected Improvement (EI). EI attempts to quantify the improvement of the parameters chosen. The average improvement is quantified by sampling the objective function at $x$. The acquisition function then computes the expected value of the improvement function at a certain $x$. EI is expressed as 
	\begin{equation*}
	\mathrm{EI}(\mathbf{x})=\left\{\begin{array}{ll}
	\left(\mu(\mathbf{x})-f\left(\mathbf{x}^{+}\right)\right) \Phi(Z)+\sigma(\mathbf{x}) \phi(Z) & \text { if } \sigma(\mathbf{x})>0 \\
	0 & \text { if } \sigma(\mathbf{x})<0,
	\end{array}\right. 
	\end{equation*}
	where $ Z = \frac{\mu(\mathbf{x}) - f(\mathbf{x}^+) - \xi }{\sigma(\mathbf{x})}$ , \cite{aktiv_bo} 
	\\\\
	The Bayesian Optimization starts by sampling from the objective function, then it fits the Gaussian Process to the sampled points and then iterates the next points in the hyperparameter space using the fitted GP and Expected Improvement. As for the implementation of Baysesian Optimization the GPyOpt library is used \cite{bo_lib}. The exploration / exploitation trade-off variable \textit{jitter} is set to the default value of 0.01.
	
	\subsection{Permutation test}
	
	
	\subsection{Bias Correction	n Methods}
	
	\section{Results}
	\textbf{TODO \\ dokumentér reproducerbarheden af resultaterne \\ tabeller og figurer over resultater \\ forklar resultaterne endten i diskussion eller resultater.}
	
	\section{Discussion} \label{discussion}
	\textbf{TODO: \\ etisk diskussion omkring teknologien der er arbejdet med}
	
	\section{Conclusion}
	\textbf{TODO: \\ Opsummér resultater og anbefalinger fra projektet \\ Må ikke indeholde noget nyt \\ Skal svare på spørgsmålene fra problemformuleringen \\ Konklusion, abstract og indledning skal give samlet billede af projektet}
	
	\section{Appendix}
	
	\begin{figure}[H]
		\centering
		\includegraphics[width=0.5\linewidth]{imgs/c_charge_degree}
		\caption{}
		\label{fig:cchargedegree}
	\end{figure}
	
	\begin{figure}[H]
		\centering
		\includegraphics[width=0.5\linewidth]{imgs/charge_degree_score}
		\caption{}
		\label{fig:chargedegreescore}
	\end{figure}
	
	\begin{thebibliography}{9}
		
		\bibitem{bo_lib} Machine Learning Group, University of Sheffield: "GPyOpt’s documentation", at \url{https://gpyopt.readthedocs.io}
		
		\bibitem{equal_of_oppor} M. Hardt, E. Price, and N. Srebro. Equality of Opportunity in Supervised Learning. In NIPS, 2016.
		
		\bibitem{Zafar} M. B. Zafar, I. Valera, M. G. Rodriguez, and K. P. Gummadi. Fairness constraints: A mechanism for fair classification.
		In ICML Workshop on Fairness, Accountability, and Transparency in Machine Learning, 2015.
		
		\bibitem{floridaLaw} The State of Florida, "The 2019 Florida Statutes", 1995,  \url{http://www.leg.state.fl.us/statutes/index.cfm?App_mode=Display_Statute&URL=0100-0199/0119/0119.html}, visited 12-03-2020
		
		\bibitem{propublicaAnalysis} ProPublica, "How We Analyzed the COMPAS Recidivism Algorithm", 2016, \url{https://www.propublica.org/article/how-we-analyzed-the-compas-recidivism-algorithm}, visited 12-03-2020
		
		\bibitem{aktiv_gp} 
		
		\bibitem{aktiv_bo} 
		
	\end{thebibliography}
	
	\newpage
	\bibliographystyle{IEEEbib}
	\bibliography{refs}
\end{document}
