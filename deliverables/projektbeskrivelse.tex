% !TeX spellcheck = da_DK
\documentclass[11pt, fleqn]{article}
%\usepackage{siunitx}
\usepackage{texfiles/SpeedyGonzales}
\usepackage{texfiles/MediocreMike}
\title{Projektbeskrivelse}
\author{Oskar Eiler Wiese Christensen s183917, Anders Henriksen s183904, Dagh}

\begin{document}
	\maketitle
	En beskrivelse på ca. 1 side. Projektbeskrivelsen skal gerne beskrive
	baggrunden for projektet ledende frem til problemformuleringen/de
	forskningsspørgsmål der ønskes besvaret. 
	
	\section*{Purpose}
	Safe AI bliver et større og større talepunkt som det begyder at gå op for for forskere at den generelle befolkning har en tendens til at stole blindt på implementeret kunstig intelligens. 
	
	\section*{Scope}
	
	
	\section*{Success criteria}
	
	
	\section*{Outcome}
	
	
	
	\section*{Forskningsspørgsmål til data}
	\begin{enumerate}
		\item Hvordan er data "skæv"? altså: hvilke bias er der i data? 
		\begin{itemize}
			\item[--] Hvordan undersøger man, om der er bias i data og hvilken bias der er?
			\item[--] Hvad betyder det, at der er et bias?
			\item[--] evt kom tilbage til at finde andre sværere metoder
		\end{itemize}
		
		\item Hvad gør det ved min algorithme, at data er skævt/at der er bias i data?
		\begin{itemize}
			\item[--] Hvad betyder det? Hvad betyder bias i en algoritme?
			\item[--] Hvordan kvantificerer jeg, at der er bias?
		\end{itemize}
		
		\item Forstå og implementere forskellige bias-metoder bl.a. metoden fra metoden fra "Equality of opportunity in supervised learning".
		
		\item Etisk diskussion
		
	\end{enumerate}
	
\end{document}