% !TeX spellcheck = en_US
\documentclass[11pt, fleqn, titlepage]{article}
\usepackage{texfiles/SpeedyGonzales}
\usepackage{texfiles/MediocreMike}
\newcommand{\so}[2]{{#1}\mathrm{e}{#2}}
\usepackage{hyperref}
\usepackage{amsmath}
\usepackage{ragged2e}
\hypersetup{
	colorlinks=true,
	linkcolor=blue,
	filecolor=magenta,      
	urlcolor=cyan,
}
\usepackage{graphicx}
\title{Feedback til Myretuen}
\author{Anders Henriksen \\ Oskar Eiler Wiese Christensen  \\ \texttt{\{s183917, s183904\}@student.dtu.dk}}
\date{\today}

\pagestyle{plain}
\fancyhf{}
\rfoot{Page \thepage{} of \pageref{LastPage}}

\begin{document}
	
	\maketitle
	
		\begin{figure}[H]
		\centering
		\includegraphics[width=0.4\linewidth]{"billeder/Screenshot from 2020-03-24 08-21-13"}
	\end{figure}
	
	\textbf{Husk at bruge burgermodellen og de tre C'er!}
	
	\textbf{Husk mere end en halv sides brødtekst og fem kritiske spørgsmål!}
	
	\textbf{Husk at der skal være tale om forslag og ikke påtvingelse!}
	
	\section{Positve}
	
	\begin{itemize}
		
		\item \textbf{Skriftlig formidling} I refererer godt til figurer (bortset fra figur 6) Hele vejen gennem teksten, hvilket hjælper med at give figurerne relevans. 
		
		\item \textbf{Forside} Forsiden er god, da den indeholder titel, navne, studienumre, data og logo som forventet.
		
		\item Sektion 1.1 i introduktion kommer der en god motivation af hvorfor reinforcement learning er spændende i forhold til den måde vi som mennekser lære på. 
		
		\item \textbf{Metoder} Metoderne er godt skrevet og er underbyyget af en masse teori blandt andet gennem matematiske formler. Der bliver dækket over de emner, som kunne være mangelsesværdig for en modtager, der har vores uddannelse men ikke har haft om dette emne før. Teksten kan dog blive en smule langtrukken. 
		
	
		
	\end{itemize}
	
	\section{Negative}
	
	
	\begin{itemize}
		\item \textbf{Titel} Måske titlen skal være på engelsk nu når rapporten er skrevet på engelsk.
			
		
		\item \textbf{Skriftlig formidling} Sørg for at holde rapporten objektiv ved at bruge passiv form i stedet for "we" og nogle gange "I".
		
		\item \textbf{Skriftlig formidling} Referencen til figur 4 burde være meget tidligere i sektionen, men dette nævner vi også med hensyn til den sektion for sig.
		
		\item \textbf{Skriftlig formidling} Der er en stor mængde stavefejl, men da teksten ikke er færdigskrevet er dette også forventet.
		
		\item Der er virkelig mange $\backslash$$\backslash$$\backslash$$\backslash$ og næsten ingen $\backslash$$\backslash$. Dette kan få teksten til at se en smule for usammenhængende ud.
		
		\item \textbf{Forside} Titlen kan godt være en smule mere specifik.
		
		\item \textbf{Abstract} Husk at skrive et abstract og evt. forord, når i når dertil. En indholdsfortegnelse med højst 2-3 indryk kunne også være en god ide.
		
		\item \textbf{Introduktion} Ifølge os kunne introduktionen godt bruge noget mere kreativitet for at hooke læseren.
		
		\item \textbf{Introduktion} Der er ikke som sådan nogen tydelig afgrænsning af emnet. Det er ikke tydeligt, hvad rapporten kommer til at handle om.
		
		\item \textbf{Introduktion} I ser umiddelbart ud til ikke at have en state of the art-sektion (dog har i skrevet om tidligere arbejde i metoder).
		
		\item \textbf{Introduktion} I mangler en problemformulering. Det er okay at skrive denne ud rent som spørgsmål da det lægger vægt på, hvad der skal svares på.
		
		\item \textbf{Introduktion} I kommer heller ikke ind på hvem der har brug for resultaterne og nogle af metoderne kunne godt blive bedre introduceret i introduktionen.
		
		\item \textbf{Introduktion} I virkeligheden kunne det være en god ide at forklare hele spillet fra starten af i stedet for at henvise til en gennemgang i bunden af introduktionen.
		
		\item \textbf{Introduktion} Mange af punkterne i introen kommer til at virke som et metodeafsnit, fordi der bliver talt om metoderne men det ikke bliver sat i en større kontekst.
		
		\item \textbf{Introduktion} Hvis i føler, at i hat brug for det, burde i lave en sektion dedikeret itl at forklare, hvad der kommer til at ske i resten af rapporten.
		
		\item \textbf{Introduktion} Introduktionen er meget kort i forhold til metode afsnitet. Der kunne sagtens skrives en større motivation samt som tidligere nævnt en mere konkret state of the art.
		
		\item \textbf{Metoder} Husk at komme ind på hvordan jeres metoder er implementeret i forhold til programmeringen, når i når dertil.
		
		\item \textbf{Metoder} I virker til mange steder i rapporten at bruge lang til på at beskrive metoder, i ikke har tænkt jer at bruge. Det kan ødelægge den røde tråd og gøre det svært at følge med i forklaringen. Et eksempel er i sektion 2.9
		
		\item \textbf{Metoder} Da dette ikke nødvendigvis er en matematik retning, ville jeg måske forslå, at nogle af udledningerne af fx update rule samt andre udledninger skal i appendix, så det virkelig står klart for læseren hvilken weight update i bruger. 
		
		\item 
		
		\item \textbf{Data/environment} Her bruger i også lang tid på at beskrive principper, i ikke har brug for.
		
		\item \textbf{Data/environment} Det ville være en ide at rykke illustrationer til toppen af 3.1 og understrege de helt grundlæggende regler og opbygningen før alt andet. Ellers er det svært at følge med i en forklaring om spillet.
		
		\item \textbf{Data/environment} Implementation er i virkeligheden ikke færdigt efter 3.3, så i kunne overveje at gøre resten af sektionerne i 3 til undersektioner af 3.3.
		
		\item \textbf{Data/environment} I bruger forskellige nomenklatur for det at angribe en anden myre, det kunne være en god ide at blive enige om.
		
		\item \textbf{Referencer} Husk kun at sætte referencer på ligninger hvis i rent faktisk refererer til dem, ellers er de ligegyldige.
		
	\end{itemize}
	
	\begin{figure}[H]
		\centering
		\includegraphics[width=0.9\linewidth]{"billeder/Screenshot from 2020-03-24 08-21-22".png}
	\end{figure}
	
	
	
	
	
\end{document}
